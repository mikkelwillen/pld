\section*{The Common Part}

\subsection*{A4C.1}
The statement for adding local variables
$$\texttt{local } \textbf{var } \texttt{in } \textit{Stat}$$
creates a new variable in the statement, and the variable is instantiated to $0$. To get the inverse of this, we need to assure that the inverse operations are performed, to ensure that tho local variable returns to the value $0$. That is, the inverse of the statement is
$$\texttt{local } \textbf{var } \texttt{in } \textit{Stat'}$$
where \textit{Stat'} is the inverse of \textit{Stat}.

\subsection*{A4C.2}
A gcd procedure does not output garbage in every iteration is shown below.

\begin{verbatim}
read m;
read n;

gcd()
  local g in
    if (n == 0)
      g += m;
      write g
    else
      g += m mod n
      m <-> n
      g <-> n
      gcd()
      g <-> n
      m <-> n
      g -= m mod n
    fi (n == 0)

gcd()

write m;
write n
\end{verbatim}

This procedure first reads m and n, then calls the function \texttt{gcd()}. This function creates a local variable \texttt{g} and then performs the same check as the repeat-loop in the procedure from figure 13.6 in "Programming Language Design and Implementation". If \texttt{n} is equal to zero, this means the greatest common divisor is the value of \texttt{m}, and therefore \texttt{g}, which is currently $0$, has \texttt{m} added to its value and is written. If \texttt{n} is not equal to zero, we perform the same operations as in the repeat-loop, then call \texttt{gcd()} recursively, and then performs the inverse operations to go back to the original values of the variables.

\subsection*{A4C.3}
The gcd program written in Python looks as follows.

\begin{verbatim}
m = int(input())
n = int(input())

def swap(x,y):
  a = x
  x = y
  y = a
  return x,y

def gcd():
  global m
  global n
  g = 0
  if (n == 0):
    g += m
    print(g)
    g = 0
    assert(n==0)
  else:
    g += m % n
    m,n = swap(m,n)
    g,n = swap(g,n)
    gcd()
    g,n = swap(g,n)
    m,n = swap(m,n)
    assert(n!=0)
  
gcd()  

print(m)
m = 0
print(n)
n = 0
\end{verbatim}

