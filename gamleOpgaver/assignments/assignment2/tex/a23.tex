\section*{A2.3}

\subsection*{a)}
A suggestion for a control structure to add to PLD LISP could be a while loop, as it makes structuring of code more intuitive, makes the program easier to understand, and is required for Turing completeness.

A suitable syntax could be \texttt{(while condition body)}, where the \textit{condition} is the part that determines if the loop should run once more, and the \textit{body} is the code that is run if the \textit{condition} is true.

The second suggestion for a control structure could be exception handling, in the form of a try-with functionality. This way you would avoid errors when giving the wrong kind of input to functions, because it would not be necessary to make a case for each type of input.

A suitable syntax could be \texttt{(trywith expression exception)}, where the \textit{expression} is the body that is performed, and the \textit{exception} is the exception handler.

\subsection*{b)}
