\section*{A2.4}

\subsection*{a)}
First we check whether \texttt{upToLength} is linear. Lets write [1,17,11,5] as [3,17]++[11,5].\\
\texttt{upToLength}([3,17]++[11,5]) returns [1,2,3,4].\\
Now lets check whether that is the same as \texttt{upToLength}([3,17])++\texttt{upToLength}([11,5]).\\
\texttt{upToLength}([3,17])++\texttt{upToLength}([11,5]) = [1,2]++[1,2] = [1,2,1,2].\\
Thus the function is not linear.


Now we check whether the function is homomorphic. Lets first find iota $\iota = upToLength([]) = []$.\begin{align*}
upToLength([3,17]++[11,5]) &= upToLength([3,17]) \oplus upToLength([11,5])\\
&= [1,2] \oplus [1,2]
\end{align*}
Thus $xs \oplus ys$ must be $xs ++ map(lambda y: length(xs) + y, ys)$.

\subsection*{b)}
First we check whether \texttt{everyOther} is linear. Lets write ['e','v','e','r','y'] as ['e','v','e']++['r','y'].\\
\texttt{everyOther}(['e','v','e']++['r','y']) returns ['e','e','y'].\\
Now lets check whether that is the same as \texttt{everyOther}(['e','v','e'])++\texttt{everyOther}(['r','y']).\\
\texttt{everyOther}(['e','v','e'])++\texttt{everyOther}(['r','y']) = ['e','e']++['r'] = ['e','e','r'].\\
Thus the function is not linear.

Now we check whether the function is homomorphic. Lets first find iota $\iota = everyOther([]) = []$.\begin{align*}
everyOther(['e','v','e']++['r','y']) &= everyOther(['e','v','e'])\oplus everyOther(['r','y'])\\
&= ['e','e']\oplus ['r']
\end{align*}
There is no $\oplus$ that will give us the right answer, thus this function is not homomorphic.

\subsection*{c)}
Again we check whether \texttt{returnEmpty} is linear.\\
\texttt{returnEmpty}([1,2]++[3,4]) = []\\
\texttt{returnEmpty}([1,2])++ \texttt{returnEmpty}([3,4]) = []\\
Thus the function is linear.

When a function is linear, it is also homomorphic. Here is why:
\begin{align*}
returnEmpty([]) &=[]\\
returnEmpty([1,2]++[3,4]) &= returnEmpty([1,2]) \oplus returnEmpty([3,4])\\
&= [] \oplus []
\end{align*}
Thus if $\oplus$ is $++$ the function is homomorphic.

\subsection*{d)}
We check if the function is linear. \texttt{appendAll}([[1],[4,2]]++[[],[1]]) = [1,4,2,1].\\
\texttt{appendAll}([[1],[4,2]])++\texttt{appendAll}([[],[1]]) = [1,4,2,1]. Thus the function is linear.

Again, if a function is linear, it is also homomorphic. \texttt{appendAll}([])=[]=$\iota$.\\
\begin{align*}
appendAll([[1],[4,2]]++[[],[1]]) &= appendAll([[1],[4,2]]) \oplus appendAll([[],[1]])\\
&= [1,4,2] \oplus [1]
\end{align*}
Thus the function is homomorphic with $\iota$=[] and $\oplus$ is $++$.

\subsection*{e)}
Again we check whether the function is linear. \texttt{sorted}([1,3,3,7]) returns [1,7].\\
\texttt{sorted}([1,3])++\texttt{sorted}([3,7]) returns [1,3,3,7]. Thus the function is not linear.

\begin{align*}
sorted([]) &= []\\
sorted([1,3]++[3,7]) &= sorted([1,3]) \oplus sorted([3,7])\\
&= [1,3] \oplus [3,7]\\
sorted([1,3]++[2,7]) &= sorted([1,3]) \oplus sorted([2,7])\\
&= [1,3] \oplus [2,7]
\end{align*}

Thus the function is homomorphic with
$$\oplus = \left\{ \begin{array}{ll}
l1[0] ++ l2[1] & if \; l1[1] \leq l2[0] \\
l1[0] & otherwise
\end{array}\right.$$
