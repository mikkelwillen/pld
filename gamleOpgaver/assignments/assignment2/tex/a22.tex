\section*{A2.2}

\subsection*{a)}
An advantage is that there will be only one value for truth. If any non-zero values are considered true, then comparing two truth values can evaluate to false, as the values are technically not equal, even though they are both true. 

Another advantage could be that it would avoid false positives, because only $1$ would be considered as \textit{true}. When there are fewer values that are considered as \textit{true}, it is easier to predict and understand how to do calculations with truth values.

Disadvantages are that more checks would have to be implemented, to be certain that the input value is either $0$ or $1$. Furthermore, not all values would be accepted in an \textit{if} or \textit{while} statement, which means that we would need a way to deal with run-time errors, as opposed to the other version, where we can always evaluate values to either \textit{true} or \textit{false}.

\subsection*{b)}
Advantages could be that an and-operation between the truth value \texttt{0xfff...ff} and an arbitrary value $x$ will always return the value $x$, and likewise an or-operation between $0$ and a value $x$ will always return $x$.

Furthermore, the negation of the truth value \texttt{0xfff...ff} is equal to $0$, and vice versa. This can be convenient when performing calculations.

A disadvantage is that multiple values evaluate to \textit{true}.

\subsection*{c)}
A suggestion to what values would be sensible for representing \textit{true} and \textit{false} could be $0$ for \textit{false} and all non-zero values for \textit{true}. The implementation of logical conjunction and disjunction could look as follows.

\begin{verbatim}
(define and (lambda (0 0) (0)
                    (0 x) (0)
                    (x 0) (0)
                    (x x) (1)
            )
)

(define or (lambda (0 0) (0)
                   (0 x) (1)
                   (x 0) (1)
                   (x x) (1)
           )
)
\end{verbatim}

Another suggestion could be \texttt{FALSE} for \textit{false} and \texttt{TRUE} for \textit{true}. The implementation of logical conjunction and disjunction could then look as follows.

\begin{verbatim}
(define and (lambda (FALSE FALSE) (FALSE)
                    (FALSE TRUE) (FALSE)
                    (TRUE FALSE) (FALSE)
                    (TRUE TRUE) (TRUE)
            )
)

(define and (lambda (FALSE FALSE) (FALSE)
                    (FALSE TRUE) (TRUE)
                    (TRUE FALSE) (TRUE)
                    (TRUE TRUE) (TRUE)
            )
)
\end{verbatim}

These operations could be done variadic too, as it could be done as a recursive version of the binary solution. A proper return value when no arguments are given could then be \textit{true}, as the recursion would then be easier to implement. Furthermore, it might make sense that \texttt{x and ()} returns \textit{true} if $x$ is true, thus $()$ should evaluate to \textit{true}.
