\section*{A 1.2}

\subsection*{a)}
Below the programs and the machine is represented as diagrams (H is short for Haskell).
\begin{center}
  \begin{picture}(230,100)(-90,-20)
    \put(-30,0){\compiler{C,ARM,ARM}}
    \put(-130,0){\compiler{H,H,C}}
    \put(80,0){\interpreter{H,C}}
    \put(180,0){\machine{ARM}}
  \end{picture}
\end{center}

\subsection*{b)}
First we want to convert the interpreter of Haskell in C to an interpreter of Haskell in ARM:
\begin{center}
  \begin{picture}(230,100)(-90,-20)
    \put(0,0){\compiler{\interpreter{H,C},\machine{ARM},\interpreter{H,ARM}}}
  \end{picture}
\end{center}
So now we have the interpreter in ARM, and so we want to interpret the compiler from Haskell to C written in Haskell so we get a compiler from Haskell to C written in ARM.
\begin{center}
  \begin{picture}(230,150)(-20,-90)
    \put(0,0){\compiler{H,\compiler{H,\interpreter{H,\machine{ARM}},\compiler{H,\compiler{C,\machine{ARM},\compiler{H,ARM,C}},C}},C}}
%    \put(0,0){\compiler{\compiler{H,H,C},\interpreter{H,\machine{ARM}},\compiler{\compiler{H,C,C},\machine{ARM},\compiler{H,ARM,C}}}}
  \end{picture}
\end{center}
After running this, we have a compiler from Haskell to C written in ARM:
\begin{center}
  \begin{picture}(230,100)(-90,-20)
    \put(0,0){\compiler{H,ARM,C}}
  \end{picture}
\end{center}

\subsection*{c)}
Now we want to convert a program \textit{p} written in Haskell to a program written in ARM, using our new compiler. This can be done on our ARM machine as shown below.
\begin{center}
  \begin{picture}(230,100)(-20, -20)
    \program{p,\compiler{H,\machine{ARM},\program{p,\compiler{C,\machine{ARM},\program{p,ARM}}}}}
  \end{picture}
\end{center}
